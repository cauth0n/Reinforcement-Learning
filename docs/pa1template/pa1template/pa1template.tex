% This template was created by Ben Mitchell
% for Dr. Sheppard's AI class, CS 335/435, spring 08.

% For those who want to learn LaTeX, this is a decent place to start:
% http://en.wikibooks.org/wiki/LaTeX
% Note that the proper pronunciation is "la tek", not "lay teks".
%
% There are lots of latex tutorials and primers online; just be careful with
% google images.

\documentclass[12pt,letterpaper]{article}

\usepackage{amsmath, amsthm, graphicx, multicol}

\title{A Title: \\ with a more informative subtitle}
\author{Author's Name Here}

\begin{document}
\maketitle

\begin{abstract}
This is where your abstract goes.  An abstract should be a short paragraph
telling the reader what you did, who might care, and why.  Your abstract is what
will be read first; most readers will use the abstract to decide whether or not
to read the rest of the paper.  It is therefore important to have an abstract
that is both clear and concise, as well as accurately describing what the paper
is about.  50-100 words is generally a good guideline for length.

For instance, for this assignment, you should mention that you used various
strategies to play the game Konane.  You should also mention
how the various strategies compared.  Overall, your abstract should read as a
very short (but still grammatically correct) summary of what the rest of the
paper is about.  It should not contain details of methods used or results;
you'll get to those later.  It should contain a brief summary of your main
conclusions and/or contributions to the field. Your abstract should not be any
longer than this one; as abstracts go, this is already too long.
\end{abstract}

\section{Introduction}
In the first section of your paper, which should be entitled ``Introduction,''
you should introduce the subject of your paper to the reader.  There are several
components to this.  You probably want to begin by talking about what problem
your work is trying to solve, and why it is a problem worth trying to solve.
Explain to your reader why they should care.  This is also a good time to
introduce basic nomenclature related to your subject.  You can assume that your
reader is a reasonably well educated individual, and has some knowledge of
computer science, but don't assume that they're an expert in your field.  You've
spent a bunch of time studying your topic (in this case, game AI), but
your reader may not have.  Take the time to bring him or her up to speed; this
is what the introduction is all about.

The introduction in a real paper is generally several pages, but yours doesn't
need to be that long for this assignment.  It should, however, be long enough to
convey the information you need to convey.  It should move from the initial, high level
overview of the topic to a more detailed description of what other related work
has been done on the topic, and any other background information needed to
understand the later parts of the paper.  For example, you should talk about the nature
of game search in general, and what other work has been
done to solve this important problem.  More specific details of the algorithms and
techniques you used should wait for later sections.

Because your introduction discusses prior related work, you should be sure to
properly cite that prior related work.  If your introduction does not have at
least a couple of citations, then it is either inadequate in scope, or it fails
to properly cite works referenced.  Don't do either.  Later sections of the
paper should also contain proper citations where needed, but the bulk of the
citations in any given paper can usually be found in the introduction, as that
is where the most related publications are discussed.

Your introduction should conclude by describing the organization of the rest of
the paper; for example:
\begin{quote}
``In section 2, we describe the theoretical basis for our algorithm, and in section
3, we describe the algorithm itself and our experiments with it.  In section 4
we present the results of our experiments, and conclude in section 5 with a
discussion of what these results mean, and possible directions for future
work.''
\end{quote}

Not all papers have this sort of organizational statement, but it can be helpful
due to the fact that the naming and organization of the sections in papers are
not entirely standard.  While almost all scientific papers begin with an
introduction, then discuss the experiments and results, and finally conclude,
the precise organization of that middle part can vary somewhat based on the
subject matter and the preferences of the author (or sometimes the journal).

\section{Game Search}
Here, you should talk in more detail about the specifics of how computer games
are implemented, as a preface to the next subsection.

\subsection{Traditional Minimax Search}
Here, talk about minimax search in game playing.  You can give the generic
algorithm for minimax, and talk about the the conditions under which minimax
is appropriate.

\subsection{Minimax with Alpha-Beta Pruning}
Here, talk about how the alpha-beta algorithm works.  Give the basic algorithm, and
discuss what needs to be done for your implementation.

\section{Experimental Methods}

In this section, you describe your experimental methodology; this is where you talk
about your data and what you did with it.  Talk about what sorts of experiments
you performed, and how you validated them.  For example, if you used self-play
or you experimented with a human playing with the computer, you would say that 
you used it, define what it is, and discuss
how you implemented it.  

\section{Results}
\begin{figure}
\begin{center}
\includegraphics[width=2.5in]{konane1.png}
\end{center}
\caption{This is a caption on the figure}
\label{somefigure}
\end{figure}

\begin{table}
\begin{center}
\begin{tabular}{|c||c|c|c|}
\hline
& col1 & col2 & col3\\
\hline \hline
row1 & a & b & c\\
\hline 
row2 & d & e & f\\
\hline 
\end{tabular}
\end{center}
\caption{This is a caption on the table}
\label{sometable}
\end{table}

The results section should contain your results.  It should \emph{not} contain
your interpretation of those results.  That comes later.  This section should be
made up primarily of graphs and tables that show your data.  You should also
have a small amount of text describing what each of the tables and graphs shows,
since the caption on the figures should be short.  Having text describing the
specifics of the experiment that led to that particular table would also be
good.  

I'm not going to tell you exactly what tables or graphs you should have here,
since it will depend a bit on your results.  You should be sure that your
results section contains sufficient data to support your conclusions about the
relative strengths and weaknesses of the different algorithms.  You should also
be sure that your data is complete; that is, don't leave data out simply because
it doesn't support the point you're trying to make.

You should also be sure that your results are clear and interpretable.  Seven
pages of raw binary data will do nothing to edify your reader.  Similarly, a
1 inch square graph with 12 lines plotted on it will be difficult to extract
meaning from, as will a graph with poor (or no) labels on the axes.  Your
results should be legible both on screen and in hard copy. By the way, do
not rely on color difference to make your points since many people read
black-and-white print outs of the papers (including your instructor).

You don't want to present results that are just raw data, since that is hard to
interpret.  But you don't want to be overly abstract, either, since that leads to
results that have little or no meaning (e.g., ``the average over all different
data sets, algorithms, and parameters'' is a completely useless statistic for
comparing algorithms).

\section{Discussion}
The discussion section is where you discuss your interpretation of the data you
presented in the results section.  This is where you tell the reader how great
your algorithm is, and how interesting it is that \emph{this} performed better
than \emph{that} on some given data set.  You can also speculate about causes
for interesting behaviors; for example, if you think you might know why it fails
so badly on some particular case, or if you have an insight into why it did well
on another case.  You don't want to be making wild guesses, but as long as you
make it clear that you are not making claims of factual proof, you can go out on
a limb a little.  For example,

\begin{quote}
``In most cases, algorithm A outperforms algorithm B with a significance of
99.8\%.  However, as can be seen from Figure \ref{somefigure}, when applied to
the E. E. Smith data set, algorithm A does no better than random chance.  It
seems likely that the failure of algorithm A to learn is due to the extremely
sparse distribution of that data set.  Because of algorithm A's heavy reliance
on data being densely sampled from the true underlying distribution, any sparse
data set is likely to show this behavior.''
\end{quote}

\section{Conclusions}
The conclusion section should be relatively short, and should not be a summary
of your paper.  It should, however, bring up what you learned and what impact
your results have on the rest of the field (and society as a
whole, if applicable).  You should conclude, and bring your paper to an  end
with any parting thoughts that are appropriate.

Certain types of papers can be ended with a ``Summary'' section instead of a
``Conclusions'' section, in which case you would, in fact, summarize the main
points of your paper.  For this paper, you should write a Conclusions section,
not a Summary.

Conclusion also often contain information about what else you would like
to do.  Sometimes this is a separate subsection, or even a section, entitled
``Future Work.''  The basic idea here is to talk about what the next steps to
take would be.  This is of benefit to others who are interested in your
work and may want to help advance it.  It is also a chance for you to
acknowledge shortcomings in your work; since we never have infinite time to
prepare a paper, there are always more experiments that would have been nice to
include.  If you list them as future work, then it at least makes it clear that
you didn't do those things because you didn't have time, rather than because you
didn't realize that they were important to do.

In your paper, you should include a brief discussion of avenues for possible
future work in your Conclusions section.  It should be tied in with the rest of
your conclusion, and should not be an unrelated section tacked on the end (or
the middle).

\section*{References}
References go here. If using \LaTeX, you probably also want to make use of {\sc Bib}\TeX. 
A sample {\sc Bib}\TeX\  file is included with this template. 
Otherwise, you will need to manage your references manually.

\bibliographystyle{plain}
\bibliography{templateBibliography}

\end{document}
